Para considerar a varição na concentração da superfície, pode-se utilizar um termo de adsorção na condição de contorno, como foi utilizado por \cite{galdikas2011modeling}, dado por $\dfrac{\alpha j}{qH_{superfície}}\left(H_0-C(0,t)\right)$, onde $j$ representa a densidade de corrente média do íon de nitrogênio, $\alpha$ é o coeficiente de adesão do nitrogênio nos componentes, $q$ é a carga elementar (equivalente a $1,602.10^{-19}C$), $H_{superfície}$ é a concentração de átomos hospedeiros na superfície (em átomos/$m^2$), $H_0$ é  concentração de átomos hospedeiros e $C(0,t)$ é a concentração de átomos de nitrogênio na superfície.

As equações da seção \ref{sec:sol-numerica-2alei} se mantêm as mesmas, exceto pelo contorno que define a superfície pela qual ocorre adsorção, que é descrito pela eq.\autoref{eq:2alei-num-plasma}

\begin{equation}
\label{eq:2alei-num-plasma}
\pdv{C(xz,t)}{t} = D(C)\pdv[2]{C(x,t)}{x} + \dfrac{\alpha j}{qH_{superfície}}\left(H_0-C(0,t)\right)
\end{equation}

As condições iniciais e a condição de contorno para $x\rightarrow\infty$ não se alteram.

Discretizando a equação \autoref{eq:2alei-num-plasma}, utilizando $j_0 = \dfrac{j}{qH_{superfície}}$ obtêm-se:

\begin{gather*}
\dfrac{C_0^{j+1} - C_0^j}{\Delta t} = D\dfrac{C_{i+1}^{j+1} - C_{i}^{j+1}}{(\Delta x)^2} + \alpha j_0\left(H_0-C(0,t)\right) 
\end{gather*}

\begin{equation}
\label{eq:2alei-num-plasma-discr}
C_0^{j+1} = C_0^j + Fo(C_{1}^{j+1} - C_{0}^{j+1})+ \alpha j_0 \Delta t \left(H_0-C_{0}^{j+1}\right)  \qquad   j \in [0,\ T]
\end{equation}

Para resolver o probelma diferencial, desenvolve-se o sistema de equações:

\begin{equation*}
\label{eq:depoisFo}
\begin{matrix}
i = 0: \qquad (1 + Fo + \alpha j_0 \Delta t)C_0^{j+1} - FoC_{1}^{j+1} & = & C_0^j + \alpha j_0 \Delta t H_0 \\
\\
i = 1: \qquad -FoC_{0}^{j+1} + (1+2Fo)C_1^{j+1} - FoC_{2}^{j+1} - C_1^j & = & 0\\
(1+2Fo)C_1^{j+1} - FoC_{2}^{j+1} - FoC_{0}^{j+1} & = & C_1^j \\ 
\\
i = 2: \qquad -FoC_{1}^{j+1} + (1+2Fo)C_2^{j+1} - FoC_{3}^{j+1} - C_2^j & = & 0\\
(1+2Fo)C_2^{j+1} - FoC_{3}^{j+1} - FoC_{1}^{j+1} & = & C_2^j \\ 
\vdots \\
i = n: \qquad -FoC_{n-1}^{j+1} + (1+2Fo)C_n^{j+1} - FoC_{n+1}^{j+1} - C_n^j & = & 0\\
(1+2Fo)C_n^{j+1} - FoC_{n-1}^{j+1} & = & C_n^j \\
\end{matrix}
\end{equation*}

Analogamente ao que foi feito na seção \ref{sec:sol-numerica-2a-lei}, considerando o sistema de equações dado por \cmrtext{AC$^j+1$ = B$^j$}, \cmrtext{A, C e B} podem ser definidos por:

\begin{equation*}
	\cmrtext{A} =
	\begin{bmatrix}
		(1 + Fo + \alpha j_0 \Delta t) & -Fo &         &        & 0\\
		-Fo & (1+2Fo) &  -Fo   &        & \\
		    & -Fo &   (1+2Fo)   & \ddots & \\
		    &     &  \ddots & \ddots & -Fo \\
	     0  &     &         &  -Fo   & (1+2Fo) 
	\end{bmatrix}
\end{equation*}
\begin{equation*}
	\cmrtext{C^{j+1}} =
	\begin{bmatrix}
		c_0^{j+1} \\
		c_1^{j+1} \\
		c_2^{j+1} \\
		\vdots \\
		c_n^{j+1}
	\end{bmatrix}
	\qquad
	\cmrtext{B^j} =	
	\begin{bmatrix}
	c_0^j + \alpha j_0 \Delta t H_0 \\
		c_1^j \\
		c_2^{j} \\
		\vdots \\
		c_n^{j}
	\end{bmatrix}
\end{equation*}

\cmrtext{A} é uma matriz tridiagonal, \cmrtext{AC$^{j+1}$ = B$^j$} é um sistema de $n$ equações lineares com $n$ incógnitas. Logo, também é possível utilizar o algoritmo de Thomas visto na seção \ref{sec:algo-thomas} para solucionar o problema.

