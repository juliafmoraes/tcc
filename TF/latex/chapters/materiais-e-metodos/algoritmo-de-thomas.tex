O Algoritmo de Thomas é um algoritmo de resolução de sistemas de equações represetados por uma matriz tridiagonal e é uma simplificação da Eliminação Gaussiana ($\frac{2}{3}O(n^3)$), com custo $O(n)$.

Um sistema de equações cujo conjunto de equações possa ser representado pela eq.(\ref{eq:thomas-sistema1}), também pode ser escrita matricialmente, na forma de matriz tridiagonal.
\begin{equation}
\label{eq:thomas-sistema1}
a_i x_{i-1} + b_i x_i + c_i x_{i+1} = d_i \qquad i = 1, ..., n
\end{equation}

\begin{equation*}
	\begin{bmatrix}
		b_1 & c_1 &         &        & 0\\
		a_2 & b_2 &   c_2   &        & \\
		    & a_3 &   b_3   & \ddots & \\
		    &     &  \ddots & \ddots & c_{n-1} \\
	     0  &     &         &  a_n   & b_n 
	\end{bmatrix}
	\begin{bmatrix}
		x_1 \\
		x_2 \\
		x_3 \\
		\vdots \\
		x_n
	\end{bmatrix}
	=\begin{bmatrix}
		d_1 \\
		d_2 \\
		d_3 \\
		\vdots \\
		d_n
	\end{bmatrix}
\end{equation*}

Com algumas operações chega-se ao algoritmo em si - seja $L_i$ a linha $i$ do sistema:

\begin{equation*}
	L_1' = L_1 \divisionsymbol b_1 \Rightarrow
	\begin{bmatrix}
		1 & \dfrac{c_1}{b_1} &         &        & 0\\
		a_2 & b_2 &   c_2   &        & \\
		    & a_3 &   b_3   & \ddots & \\
		    &     &  \ddots & \ddots & c_{n-1} \\
	     0  &     &         &  a_n   & b_n 
	\end{bmatrix}
	\begin{bmatrix}
		x_1 \\
		x_2 \\
		x_3 \\
		\vdots \\
		x_n
	\end{bmatrix}
	=\begin{bmatrix}
		\dfrac{d_1}{b_1} \\
		d_2 \\
		d_3 \\
		\vdots \\
		d_n
	\end{bmatrix}
\end{equation*}

\begin{equation*}
	L_2' = L_2 - a_2L_1'  \Rightarrow
	\begin{bmatrix}
		1 & \dfrac{c_1}{b_1} &         &        & 0\\
		0 & b_2 - a_2\dfrac{c_1}{b_1} &   c_2   &        & \\
		    & a_3 &   b_3   & \ddots & \\
		    &     &  \ddots & \ddots & c_{n-1} \\
	     0  &     &         &  a_n   & b_n 
	\end{bmatrix}
	\begin{bmatrix}
		x_1 \\
		x_2 \\
		x_3 \\
		\vdots \\
		x_n
	\end{bmatrix}
	=\begin{bmatrix}
		\dfrac{d_1}{b_1} \\
		d_2 - a_2\dfrac{d_1}{b_1}\\
		d_3 \\
		\vdots \\
		d_n
	\end{bmatrix}
\end{equation*}

Seja $c_1' = \dfrac{c_1}{b_1}$ e $d_1' = \dfrac{d_1}{b_1}$:
\begin{equation*}
	L_2'' = L_2'\divisionsymbol (b_2-a_2c_1')  \Rightarrow
	\begin{bmatrix}
		1 & c_1' &         &        & 0\\
		0 & 1 &   \dfrac{c_2}{b_2-a_2c_1'}   &        & \\
		    & a_3 &   b_3   & \ddots & \\
		    &     &  \ddots & \ddots & c_{n-1} \\
	     0  &     &         &  a_n   & b_n 
	\end{bmatrix}
	\begin{bmatrix}
		x_1 \\
		x_2 \\
		x_3 \\
		\vdots \\
		x_n
	\end{bmatrix}
	=\begin{bmatrix}
		d_1' \\
		\dfrac{d_2 - a_2d_1'}{b_2-a_2c_1'}\\
		d_3 \\
		\vdots \\
		d_n
	\end{bmatrix}
\end{equation*}

De forma análoga para as outras linhas, chega-se que:

\begin{center}
	\begin{math}
		c_i' = 
		\left\{
    		\begin{array}{l}
      			\dfrac{c_i}{b_i}  \\
				\dfrac{c_i}{b_i - a_i c_{i-1}'}
			\end{array}
    		\begin{array}{l}
      		\qquad \\
			\qquad \\
			\qquad	   
    		\end{array}
    		\begin{array}{l}
    			i = 1 \\
      			i = 2, ... , n-1
    		\end{array}
		\right.	
	\end{math}
\end{center} 
\begin{center}
	\begin{math}
		d_i' = 
		\left\{
    		\begin{array}{l}
      			\dfrac{d_i}{b_i}  \\
				\dfrac{d_i - a_i d_{i-1}'}{b_i - a_i c_{i-1}'}
			\end{array}
    		\begin{array}{l}
      			\qquad \\
				\qquad \\
				\qquad	   
    		\end{array}
    		\begin{array}{l}
    			i = 1 \\
      			i = 2, ... , n
    		\end{array}
		\right.
	\end{math}
\end{center} 

Dessa forma, a solução para o sistema de equações pode ser obtida encontrando $x_n$, $x_{n-1}$, $x_{n-2}$, ..., até $x_1$ utilizando:

\begin{gather*}
	x_n = d_n' \;,\\
	x_i = d_i' - c_i' x_{i+1} \ para \ i = 1, ..., n-1 \;.
\end{gather*}