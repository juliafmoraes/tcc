\subsubsection{Nitretação Gasosa}
\label{sec:nit-gas}
Para o caso de nitretação gasosa, segundo o artigo de \cite{christiansen2008nitrogen}, uma forma de simular a concentração na superfície é utilizando a relação dada pela eq.(\ref{eq:cs-gas}). Na qual t é o instante de tempo, $\beta$ determina a velocidade com que a concentração na superfície atinge a concentração em equilíbrio e $C_{eq}$ é a concentração de equilíbrio entre o gás e o aço.

\begin{equation}
\label{eq:cs-gas}
C_{s}(t) = C_{eq}(1 - e^{-\beta t}) \;.
\end{equation}

Sendo assim, basta substituir $C_s^{j+1} = C_{eq}(1 - e^{-\beta j\Delta t})$ no desenvolvimento realizado na Seção \autoref{sec:sol-numerica-2alei}, de forma que o sistema apresentado (\cmrtext{AC$^{j+1}$ = B$^j$}) tenha uma equação para $c_0$:

\begin{equation*}
	\cmrtext{A} =
	\begin{bmatrix}
		  1 & 0 &         &        & 0\\
		-Fo & (1+2Fo) &  -Fo   &        & \\
		    & -Fo &   (1+2Fo)   & \ddots & \\
		    &     &  \ddots & \ddots & -Fo \\
	     0  &     &         &  -Fo   & (1+2Fo) 
	\end{bmatrix}
\end{equation*}
\begin{equation*}
	\cmrtext{C^{j+1}} =
	\begin{bmatrix}
		c_0^{j+1} \\
		c_1^{j+1} \\
		c_2^{j+1} \\
		\vdots \\
		c_n^{j+1}
	\end{bmatrix}
	\qquad
	\cmrtext{B^j} =	
	\begin{bmatrix}
		C_{eq}(1 - e^{-\beta j\Delta t}) \\
		c_1^j \\
		c_2^{j} \\
		\vdots \\
		c_n^{j}
	\end{bmatrix}
\end{equation*}

Assim como os outros sistemas tridiagonais obtidos, pode ser resolvido utilizando o Algoritmo de Thomas.