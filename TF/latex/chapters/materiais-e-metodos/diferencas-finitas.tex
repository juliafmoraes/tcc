\par Uma forma de encontrar a solução de equações diferenciais é pelo método das diferenças finitas. A ideia fundamental desse método é aproximar a derivada de uma função ao redor de um dado ponto utilizando valores desta função em pontos vizinhos, criando uma malha de nós sobre o domínio da função.
\par Considerando um problema de equações diferenciais com condições de Dirichlet (aonde são conhecidos os valores nas extremidadades do domínio) representada pelo sistema $(S)$, podemos discretizar o intervalo [a, b] no pontos $x_i = a + ih$ para $i \in [0, n+1]$, sendo $h = \dfrac{b-a}{n+1}$, a distância entre os nós.

\begin{center}
	\begin{math}
		(S)
		\left\{
    		\begin{array}{l}
      			-u'' + p(x)u + q(x)u = g(x) \\
				u(a) = \alpha\\
				u(b) = \beta      			
    		\end{array}
    		\begin{array}{l}
      		\qquad \\
			\qquad \\
			\qquad	   
    		\end{array}
    		\begin{array}{l}
      			x \in [a, b] \\
				p, q, g \in C^0([a, b])\\
				q \geq 0 \ em \ [a, b]
    		\end{array}
		\right.
	\end{math}
\end{center} 

\par Seja $u_i$ o valor aproximado de $u(x_i)$ que deseja-se calcular e $U = (u_1, ..., u_n)^T$, $u_0 = alpha$ e $u_{n+1} = \beta$. Aproxima-se $u''(x_i)$ e $u'(x_i)$ utilizando a série de Taylor de u em torno de $x_i$.
\par A série de Taylor é a representação de uma função na forma de uma soma infinita cujos termos são calculados pelas derivadas da função em um único ponto.
A eq.(\ref{eq:taylor}) é a série de Taylor da funçao $f(x)$ em torno de $x_i$.

\begin{equation} \label{eq:taylor}
f(x_i) = \sum_{n=0}^\infty \dfrac{f^{(n)}(x_i)}{n!}(x-x_i)^n = f(x_i) + \dfrac{f'(x_i)(x-x_i)}{1!} + \dfrac{f''(x_i)(x-x_i)^2}{2!} + ...
\end{equation}

Pode-se escrever a série de Taylor de $u$ em torno de $x_i$ como visto na equação \ref{eq:taylor-u}.

\begin{equation} \label{eq:taylor-u}
u(x) = u(x_i) + u'(x_i)(x-x_i) + \dfrac{u''(x_i)}{2!}(x-x_i)^2 + ...
\end{equation}

As expansões de Taylor para $u$ em torno e $x_i$, para $x=x_{i-1}$ e $x=x_{i+1}$ estão descritas nas equações \ref{eq:taylor-u-1} e \ref{eq:taylor-u+1}, respectivamente.

\begin{equation} \label{eq:taylor-u-1}
u(x_{i-1}) = u(x_i-h) = u(x_i) - hu'(x_i) + \dfrac{h^2u''(x_i)}{2!} - \dfrac{h^3 u^{(3)}(x_i)}{3!} + O(h^4)
\end{equation}


\begin{equation} \label{eq:taylor-u+1}
u(x_{i+1}) = u(x_i+h) = u(x_i) + hu'(x_i) + \dfrac{h^2u''(x_i)}{2!} + \dfrac{h^3 u^{(3)}(x_i)}{3!} + O(h^4)
\end{equation}

Subtraindo a eq.(\ref{eq:taylor-u-1}) da eq.(\ref{eq:taylor-u+1}), obtém-se:

\begin{equation} \label{eq:u-linha}
\dfrac{u(x_{i+1}) - u(x_{i-1)}}{2h} = u'(x_i) + O(h^2)
\end{equation}

E somando as mesmas equações:

\begin{equation} \label{eq:u-linha-linha}
\dfrac{u(x_{i+1}) -2u(x_i) + u(x_{i-1)}}{h^2} = u''(x_i) + O(h^2)
\end{equation}

Subsitituindo as equações \ref{eq:u-linha} e \ref{eq:u-linha-linha} em $(S)$, obtém-se:

\begin{equation} \label{eq:u-linha-linha}
\dfrac{u(x_{i+1}) -2u(x_i) + u(x_{i-1)}}{h^2} + p(x_i)\dfrac{u(x_{i+1}) - u(x_{i-1)}}{2h}  + q(x_i)u(x_i) - g(x_i) = e_i
\end{equation}

para todo $i=1, ..., n$ com $e_i = O(h^2)$ chamado de erro de truncamento.
 
\par Logo, utilizando a notação $p_i = p(x_i)$, $q_i = q(x_i)$, $g_i = g(x_i)$, o probelma $(S)$ pode ser reescrito como:

\begin{center}
	\begin{math}
		(S)
		\left\{
    		\begin{array}{l}
      			\dfrac{-u_{i+1}+ 2u_i - u_{i-1}}{h^2} + p_i\dfrac{u_{i+1} - u_{i-1}}{2h}  + q_i u_i - g_i = 0 \\
				u_0 = \alpha\\
				u_{n+1} = \beta      			
    		\end{array}
    		\begin{array}{l}
      		\qquad \\
			\qquad \\
			\qquad	   
    		\end{array}
    		\begin{array}{l}
      			i = 1, ... , n 
    		\end{array}
		\right.
	\end{math}
\end{center} 

O problema dado por $(S)$ consiste de um sistema de $n$ equações lineares com $n$ incógnitas que pode ser escrito de forma matricial: \cmrtext{AU=B}, com \cmrtext{A=$M_n$($\mathbb{R}$)}, \cmrtext{U=($u_1, u_2, ... , u_n)^T$} e \cmrtext{B $\in \mathbb{R}^n$}.


\begin{equation*}
	\cmrtext{A} =
	\begin{bmatrix}
		2+h^2q_1 & -1+\dfrac{h}{2}p_1 & \ldots  & \ldots & \ldots & 0 \\
		-1+\dfrac{h}{2}p_2 & 2+h^2q_2 & -1+\dfrac{h}{2}p_2 &  & & \vdots\\
		\vdots &\ddots&\ddots&\ddots&& \vdots \\
		\vdots &&\ddots&\ddots&\ddots& \vdots \\
		\vdots & & & \ddots & \ddots & -1+\dfrac{h}{2}p_{n-1} \\
		0 & \ldots & \ldots & \ldots & -1+\dfrac{h}{2}p_n & 2+h^2q_n	
	\end{bmatrix}
\end{equation*}

\begin{equation*}
	\cmrtext{U} =
	\begin{bmatrix}
		u_1 \\
		u_2 \\
		\vdots \\
		\vdots \\
		\vdots \\
		u_{n-1} \\
		u_n
	\end{bmatrix}
	\qquad
	\cmrtext{B} =
	\begin{bmatrix}
		h^2g_1 + \alpha(1+\dfrac{h}{2}p_1) \\
		h^2g_2 \\
		\vdots \\
		\vdots \\
		\vdots \\
		h^2g_{n-1} \\
		h^2g_{n} + \beta(1+\dfrac{h}{2}p_n)
	\end{bmatrix}
\end{equation*}

Esse sistema de equações pode ser resolvido utilizando o algoritmo da matriz tridiagonal (algoritmo de Thomas), apresentado na seção \ref{sec:algo-thomas}.

O método das diferenças finitas amplifica erros causados pelos truncamentos e arredondamentos a cada passo de tempo. Para identificar instabilidade escreve-se a equação para um nó de forma que ao seu lado esquerdo esteja somente a incógnita para a variável dependente daquele nó, se algum coeficiente das variáveis dependentes ao seu lado direito for negativo, é instável. O método explícito é semi-instável, mas como foi utilizado o método implícito, será demonstrado que ele é incodicionalmente estável na seção \ref{sec:sol-numerica-2alei} 