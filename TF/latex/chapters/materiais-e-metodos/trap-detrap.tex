\subsection{Desenvolvimento do Modelo}
\label{sec:trap-detrap-desenvolvimento}
Como descrito na seção \autoref{sec:trap-detrap}, o modelo de \textit{trapping-detrapping} considera que os átomos de Cromo em solução sólida nos aços agem como sítios de aprisionamento de Nitrogênio. Isso ocorre devido à alta afinidade ente os dois átomos.

Para considerar esse fenômeno foi adicionado um termo adicional à equação da Segunda Lei de Fick (eq.\autoref{eq:2alei-num}), $\pdv{N_{trap}(x,t)}{t}$ que permite considerar os efeitos do aprisionamento. 

Os principais artigos que auxiliaram na elaboração desse modelo foram:
\cite{moskalioviene2011modeling}, \cite{parascandola2000nitrogen}, \cite{moller1983pidat}, \cite{peng2018numerical} e \cite{moller2001surface}.  

Nessa seção será utilizada a seguinte notação: $N_{dif}$ corresponde à concentração de nitrogênio disponível para difusão (não aprisionado), $N_{trap}$ será a concentração de nitrogênio no estado aprisionado e $N$ é a concentração total de nitrogênio em solução.

Segundo o artigo de Parascandola, o transporte de nitrogênio pode ser descrito pelo seguinte conjunto de equações:

\begin{equation}
\label{eq:trap-detrap}
\pdv{N_{dif}(x,t)}{t} = D\pdv[2]{N_{dif}(x,t)}{x} - \pdv{N_{trap}(x,t)}{t} 
\end{equation}

\begin{equation}
\label{eq:trap-detrap2}
\pdv{N_{trap}(x,t)}{t} = k_t\left[N_{dif}(x,t)\left(H_t-N_{trap}(x,t)\right)\right] - k_dN_{trap}(x,t)
\end{equation}

Aonde $k_t$ é o coeficiente de \textit{trapping} e $k_d$ é o coeficiente de \textit{detrapping}. Na difusão assistida por \textit{trapping} e \textit{detrapping}, $k_t$ é proporcional ao coeficiente de difusão ${D=D_0e^{\frac{-E_A}{k_BT}}}$ e $k_d$ é proprocional a ${e^{\frac{-E_B}{k_BT}}}$. Ainda segundo Parascandola, os coeficiente $k_t$ e $k_d$ tem pouca influência na difusão não estacionária e o aprisionamento segue uma cinética de primeira ordem, ou seja, a taxa de aprisionamento é proporcional ao produto da concentração de nitrogênio e da concentração de sítios de aprisionamento livres (não ocupados). A eq.\autoref{eq:trap-detrap2} modela o processo de aprisionamento, permitindo calcular a concentração de nitrogênio ocupando sítios de \textit{trap}. A expressão $k_t\left[N_{dif}(x,t)\left(H_t-N_{trap}(x,t)\right)\right]$ descreve o processo de \textit{trapping} denNitrogênio livre na matriz, sendo $H_t - N_{trap}(x,t)$ a concentração de traps disponíveis (inocupados) e a expressão $ k_dN_{trap}(x,t)$ descreve o nitrogênio aprisionado pelos \textit{traps} de cromo que podem se manter aprisionados ou serem liberado para a matriz, dependendo de fatores como a energia de ativação.


As hipóteses tomadas para a construção do modelo são: existe apenas um tipo de aprisionamento causado pelos átomos de cromo, cada sítio de aprisionamento pode reter apenas um átomo de nitrogênio, a concentração de sítios de aprisionamento é constante no tempo e espaço, a energia de ativação para  \textit{detrapping} não depende da fração de traps ocupados, existe equilíbrio local entre o nitrogênio nos sítios de difusão e aqueles nos sítios de aprisionamento e o fenômeno de \textit{trapping-detrapping} é controlado por difusão e segue cinética de primeira ordem.

O primeiro termo da equação\autoref{eq:trap-detrap} $\left(D\pdv[2]{N_{dif}(x,t)}{x}\right)$ se refere à difusão do nitrogênio da matriz de aço, como visto na Segunda Lei de Fick. Essa equação, juntamente com as equações \autoref{eq:trap-detrap2} e \autoref{eq:trap-detrap3} formam um conjunto de equações que descrevem o transporte de massa do nitrogênio considerando o mecanismo estudado.

A última equação desse conjunto apenas relaciona as três concentrações de Nitrogênio, mostrando que a variação da concentração total é a soma das variações do átomos disponível para a difusão e a variação de átomos aprisionados. 

\begin{equation}
\label{eq:trap-detrap3}
\pdv{N(x,t)}{t} = \pdv{N_{dif}(x,t)}{t} + \pdv{N_{trap}(x,t)}{t}
\end{equation}

Para $k_t = 4{\pi}R_tD$ e $k_d = 4{\pi}R_tDH_0$, a equação \autoref{eq:trap-detrap2} apresentada por \cite{parascandola2000nitrogen} se iguala a equação apresentada por \cite{moskalioviene2011modeling} para a taxa de aprisionamento de nitrogênio, como mostra a equação \autoref{eq:trap-detrap2-galdikas}.

\begin{equation}
\label{eq:trap-detrap2-galdikas}
\pdv{N_{trap}(x,t)}{t} = K\left[N_{dif}(x,t)\left(H_t-N_{trap}(x,t)\right) - H_0 N_{trap}(x,t)e^{\frac{-E_B}{k_BT}}\right] 
\end{equation}

Segue o que representa cada termo da equação \autoref{eq:trap-detrap2-galdikas}:

$K=4{\pi}R_tD$

$R_t$ : raio de aprisionamento de um único sítio de aprisionamento

$D=D_0e^{\frac{-E_A}{k_BT}}$ : coeficiente de difusão ($D_0$ - fator pré-exponencial de difusão)

$E_A$ : energia de ativação de difusão

$k_B$: constante de Boltzmann 
	
$T$: temperatura

$H_t$: concentração de sítios de aprisionamento

$H_0$: concentração de átomos hospedeiros
	
$E_B$ : energia de ativação para \textit{detrapping}

Para efeito desse trabalho, com o objetivo de desenvolver um modelo simples, foram considerados os seguintes valores para $k_t$ e $k_d$:
\begin{center}
$k_t = R_tD = R_tD_0e^{\frac{-E_A}{k_BT}}$ \\ $k_d = R_tDH_0 = R_tR_tD_0e^{\frac{-E_A}{k_BT}}H_0$
\end{center}

Para os quais a equação \ref{eq:trap-detrap2} pode ser escrita como:

\begin{equation}
\label{eq:trap-detrap2-me}
\pdv{N_{trap}(x,t)}{t} = K\left[N_{dif}(x,t)\left(H_t-N_{trap}(x,t)\right) - H_0 N_{trap}(x,t)e^{\frac{-E_B}{k_BT}}\right] 
\end{equation}

Com $K=R_tD$

\subsection{Condições de Contorno - Nitretação Gasosa}
\label{sec:trap-detrap-gas-cc}
Utilizando a condição de contorno para nitretação gasosa vista na Seção \ref{sec:sol-numerica-2alei2}, considerando que a concentração da superfície é a soma da concentração de nitrogênio livre com a concentração de nitrogênio aprisionado, pode-se escrever as seguintes equações:

\begin{equation}
\label{eq:trap-detrap-gas-cc1}
N(0,t) = C_{eq}(1 - e^{-\beta t})
\end{equation}

\begin{equation}
\label{eq:trap-detrap-gas-cc2}
\pdv{N_{trap}(0,t)}{t} = K\left[N_{dif}(0,t)\left(H_t-N_{trap}(0,t)\right) - H_0 N_{trap}(0,t)e^{\frac{-E_B}{k_BT}}\right] 
\end{equation}

\begin{equation}
\label{eq:trap-detrap-gas-cc3}
\pdv{N(0,t)}{t} = \pdv{N_{dif}(0,t)}{t} + \pdv{N_{trap}(0,t)}{t}
\end{equation}

Que podem ser resolvidas numericante aplicando diferenças finitas.

\subsection{Condições de Contorno - Nitretação à Plasma}
\label{sec:trap-detrap-plasma-cc}
Considerando o fluxo de difusão unidirecional, apenas do exterior para o interior do sólido, as condições de contorno podem ser expressas pelo conjunto de equações visto na Seção \ref{sec:trap-detrap-gas-cc}, substituindo a eq. \ref{eq:trap-detrap-gas-cc1} pela eq. \ref{eq:trap-detrap-cc1}

\begin{equation}
\label{eq:trap-detrap-cc1}
\pdv{N_{dif}(0,t)}{t} = -D\pdv[2]{N_{dif}(0,t)}{x} - \pdv{N_{trap}(0,t)}{t}  + j_{0}\left(H_0-N_{dif}(0,t)-N_{trap}\left(0,t\right)\right)
\end{equation}


A equação\autoref{eq:trap-detrap-cc1} se refere à difusão de Nitrogênio na interface, na qual o termo dado por $\alpha j_{0}\left(H_0-N_{dif}(0,t)-N_{trap}\left(0,t\right)\right)$ descreve a adsorção de nitrogênio, e  $j_0$ é o fluxo incidente de átomos de Nitrogênio, previamente explicado na Seção \autoref{sec:sol-numerica-2alei2}

