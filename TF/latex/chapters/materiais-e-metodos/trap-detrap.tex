Como descrito na seção \autoref{sec:trap-detrap}, o modelo de \textit{trapping-detrapping} considera que os átomos de Cromo em solução sólida nos aços agem como sítios de aprisionamento de Nitrogênio. Isso ocorre devido à alta afinidade ente os dois átomos.

Para considerar esse fenômeno foi adicionado um termo adicional à equação da Segunda Lei de Fick (eq.\autoref{eq:2alei-num}), $S(x)$ que permite considerar os efeitos do aprisionamento. Essa abordagem é uma adaptação do modelo criado em \cite{moskalioviene2011modeling}.

Nessa seção será utilizada a seguinte notação: $N_{dif}$ corresponde à concentração de nitrogênio disponível para difusão (não aprisionado), $N_{trap}$ será a concentração de nitrogênio no estado aprisionado e $N$ é a concentração total de nitrogênio em solução.

\begin{equation}
\label{eq:trap-detrap}
\pdv{N_{dif}(x,t)}{t} = D\pdv[2]{N_{dif}(x,t)}{x} - S(x,t)
\end{equation}

A equação \autoref{eq:trap-detrap}, juntamente com as equações \autoref{eq:trap-detrap2} e \autoref{eq:trap-detrap3} formam um conjunto de equações que descrevem o transporte de massa do nitrogênio considerando o mecanismo estudado.

\begin{equation}
\label{eq:trap-detrap2}
\pdv{N_{trap}(x,t)}{t} = S(x,t) = K\left[N_{dif}(x,t)\left(H_t-N_{trap}(x,t)\right) -N_0 N_{trap}(x,t)e^\dfrac{E_B}{k_BT}\right] 
\end{equation}

Segue o que representa cada termo da equação \autoref{eq:trap-detrap2}:

$K=4{\pi}R_tD$ : representa uma área (com raio $R_t$) de aprisionamento de um único sítio de aprisionamento

$D=D_0e^\dfrac{E_B}{k_BT}$
	$k_B$: constante de Boltzmann \myworries{(colocar o valor e a ref)}
	
	$T$: temperatura
	
\begin{equation}
\label{eq:trap-detrap3}
\pdv{N(x,t)}{t} = \pdv{N_{dif}(x,t)}{t} + \pdv{N_{trap}(x,t)}{t}
\end{equation}



