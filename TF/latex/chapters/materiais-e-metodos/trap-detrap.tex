Como descrito na seção \autoref{sec:trap-detrap}, o modelo de \textit{trapping-detrapping} considera que os átomos de Cromo em solução sólida nos aços agem como sítios de aprisionamento de Nitrogênio. Isso ocorre devido à alta afinidade ente os dois átomos.

Para considerar esse fenômeno foi adicionado um termo adicional à equação da Segunda Lei de Fick (eq.\autoref{eq:2alei-num}), $\pdv{N_{trap}(x,t)}{t}$ que permite considerar os efeitos do aprisionamento. Essa abordagem é uma adaptação do modelo criado em \cite{moskalioviene2011modeling}.

Nessa seção será utilizada a seguinte notação: $N_{dif}$ corresponde à concentração de nitrogênio disponível para difusão (não aprisionado), $N_{trap}$ será a concentração de nitrogênio no estado aprisionado e $N$ é a concentração total de nitrogênio em solução.

\begin{equation}
\label{eq:trap-detrap}
\pdv{N_{dif}(x,t)}{t} = D\pdv[2]{N_{dif}(x,t)}{x} - \pdv{N_{trap}(x,t)}{t} 
\end{equation}

O primeiro termo da equação\autoref{eq:trap-detrap} $\left(D\pdv[2]{N_{dif}(x,t)}{x}\right)$ se refere à difusão do nitrogênio da matriz de aço, como visto na Segunda Lei de Fick. Essa equação, juntamente com as equações \autoref{eq:trap-detrap2} e \autoref{eq:trap-detrap3} formam um conjunto de equações que descrevem o transporte de massa do nitrogênio considerando o mecanismo estudado.

\begin{equation}
\label{eq:trap-detrap2}
\pdv{N_{trap}(x,t)}{t} = K\left[N_{dif}(x,t)\left(H_t-N_{trap}(x,t)\right) - N_0 N_{trap}(x,t)e^{\frac{-E_B}{k_BT}}\right] 
\end{equation}

Segue o que representa cada termo da equação \autoref{eq:trap-detrap2}:

$K=4{\pi}R_tD$ : representa uma área (com raio $R_t$) de aprisionamento de um único sítio de aprisionamento

$D=D_0e^{\frac{-E_A}{k_BT}}$ : coeficiente de difusão \myworries{ $D_0$ - fator pre-exponencial de difusao}

$E_A$ : energia de ativação de difusão

$k_B$: constante de Boltzmann \myworries{(colocar o valor e a ref)}
	
$T$: temperatura

$H_t$: concentração de sítios de aprisionamento

$N_0$: concentração de átomos hospedeiros
	
$E_B$ : energia de ativação para \textit{detrapping}

A eq.\autoref{eq:trap-detrap2} modela o processo de aprisionamento, permitindo calcular a concentração de Nitrogênio ocupando sítios de \textit{trap}. A expressão $N_{dif}(x,t)\left(H_t-N_{trap}(x,t)\right)$ descreve o processo de \textit{trapping} de Nitrogênio livre na matriz, sendo $H_t - N_{trap}(x,t)$ a concentração de traps disponíveis (inocupados) e a expressão $ N_0 N_{trap}(x,t)e^{\frac{-E_B}{k_BT}}$ descreve o Nitrogênio aprisionado pelos \textit{traps} de Cromo que podem se manter aprisionados ou serem liberado para a matriz, dependendo de fatores como a energia de ativação.

A última equação desse conjunto apenas relaciona as três concentrações de Nitrogênio, mostrando que a variação da concentração total é a soma das variações do átomos disponível para a difusão e a variação de átomos aprisionados. 

\begin{equation}
\label{eq:trap-detrap3}
\pdv{N(x,t)}{t} = \pdv{N_{dif}(x,t)}{t} + \pdv{N_{trap}(x,t)}{t}
\end{equation}

Considerando o fluxo de difusão unidirecional, apenas do exterior para o interior do sólido, as condições de contorno podem ser expressas pelas equações:

\begin{equation}
\label{eq:trap-detrap-cc1}
\pdv{N_{dif}(0,t)}{t} = -D\pdv[2]{N_{dif}(0,t)}{x} - \pdv{N_{trap}(0,t)}{t}  +\alpha j_{0}\left(N_0-N_{dif}(0,t)-N_{trap}\left(0,t\right)\right)
\end{equation}

\begin{equation}
\label{eq:trap-detrap-cc2}
\pdv{N_{trap}(0,t)}{t} = K\left[N_{dif}(0,t)\left(H_t-N_{trap}(0,t)\right) - N_0 N_{trap}(0,t)e^{\frac{-E_B}{k_BT}}\right] 
\end{equation}

\begin{equation}
\label{eq:trap-detrap-cc3}
\pdv{N(0,t)}{t} = \pdv{N_{dif}(0,t)}{t} + \pdv{N_{trap}(0,t)}{t}
\end{equation}

A equação\autoref{eq:trap-detrap-cc1} se refere à difusão de Nitrogênio na interface, na qual o termo dado por $\alpha j_{0}\left(N_0-N_{dif}(0,t)-N_{trap}\left(0,t\right)\right)$ descreve a adsorção de nitrogênio, $\alpha$ é o coeficiente de aderência de nitrogênio para os componentes correspondentes \myworries{?????} e  $j_0$ é o fluxo incidente de átomos de Nitrogênio.
