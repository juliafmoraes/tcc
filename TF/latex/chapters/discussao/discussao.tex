Os principais fatores levados em consideração nos estudos da difusão do nitrogênio nos aços são a afinidade do cromo (que provoca o fenômeno de \textit{trapping-detrapping}, as tensões residuais, os efeitos de sputtering do processo à plasma, efeitos das reações na superfície e a dependência do coeficiente de difusão com a concentração.

Dessa forma o modelo desenvolvido possui algumas limitações, seguem algumas delas. Ele não pode ser utilizado para processos de nitretação com temperaturas superiores à 450°C, pois para tais temperaturas existe a possibilidade de precipitação de nitretos de cromo que não estão previstas no modelo. Não foram levadas em consideração os efeitos da taxa de sputtering e do surgimento de defeitos causados pelo bombardeamento de íon de alta energia no processo à plasma ou similiares. Não foram consideradas possíveis reações na superfície durante os processos de nitretação. Não foi analisado a influência de tensões causadas pela expansão do reticulado durante a incorporação de nitrogênio e não considerou-se a influência da composição no coeficiente de difusão.
