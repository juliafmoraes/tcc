Uma questão a ser levantada é a diferença nos resultados experimentais dos artigos utilizados para validação dos modelos. No experimento de Galdikas e Moskalioviene, para uma duração de 2 horas a profundidade da camada nitretada é de aproximadamente   12 $\mu m$, enquanto para Somers essa profundidade só é obtida após 23 horas. A principal diferença entre os processos é que o primeiro utilizou nitretação a plasma enquanto o segundo utilizou nitretação gasosa.  Ainda assim, outros artigos para aços com composição similares também não chegam próximo ao resultado obtido por Galdikas e Moskalioviene em \cite{moskalioviene2011modeling}. Em \cite{galdikas2011modeling}, uma profundidade similar foi obtida para 4 horas de experimento. Esse fato pode indicar uma influência do processo, que deve ser adicionada às condições de contorno do modelo e possivelmente pode causar grandes diferenças nos resultados

Os principais fatores levados em consideração nos estudos da difusão do nitrogênio nos aços são a afinidade do cromo (que provoca o fenômeno de \textit{trapping-detrapping}), as tensões residuais, os efeitos de sputtering do processo a plasma, efeitos das reações na superfície e a dependência do coeficiente de difusão com a concentração.

Dessa forma o modelo desenvolvido possui algumas limitações, das quais algumas serão mencionadas a seguir. Ele não pode ser utilizado para processos de nitretação com temperaturas superiores à 450°C, pois para tais temperaturas existe a possibilidade de precipitação de nitretos de cromo que não estão previstas no modelo. Não foram levadas em consideração os efeitos da taxa de sputtering e do surgimento de defeitos causados pelo bombardeamento de íon de alta energia no processo a plasma ou similiares. Não foram consideradas possíveis reações na superfície durante os processos de nitretação. Não foi analisado a influência de tensões causadas pela expansão do reticulado durante a incorporação de nitrogênio e não considerou-se a influência da composição no coeficiente de difusão.

Difusão nos contornos de grão pode ser desconsiderada pois a concentração de nitrogênio é maior dentro dos grãos do que no contorno segundo \cite{parascandola2000nitrogen}.

O fluxo de átomos na superfície depende de muitos fatores e é influenciado pelo tipo de processo e todas as variáveis nele envolvidas, como temperatura, pressão, potencial, densidade de corrente, etc.
A primeira simplificação feita nesse estudo foi que a concentração na superfície se mantinha constante do início ao fim do processo de nitretação. Nesse caso, considera-se que existe equilíbrio termodinâmico entre a  mistura gasosa e o metal em tratamento. Em seguida, utilizou-se uma condição de contorno que simulava a  tendência ao equílibrio termodinâmico e por último buscou-se simular o fluxo de um processo de nitretação a plasma, que não rendeu resultado satisfatório.
 
Como alternativa ao modelamento da nitretação a plasma, poderia ser utilizado uma condição de contorno utilizando a fração volumétrica da mistura gasosa do processo, como visto em \cite{garzon2006modelamento} e similarmente em \cite{franco2003numerical}. Nesses artigos o fluxo de nitrogênio na superfície é dado pela eq.\ref{eq:alternativa-cc}. Na qual $\%N_2$, corresponde à concentração de nitrogênio na superfície do metal, $\rho$ é a densidade do metal, $A$ é uma constante determinada pelas reações diretas e $B$ é uma constante determinada pelas reações inversas.

\begin{equation} \label{eq:alternativa-cc}
 J = \left[-D\pdv{N}{x}\right]_{x=0} = \dfrac{A}{\rho} - \dfrac{B}{\rho} \%N_2 
\end{equation}

O proceso de \textit{sputtering} utilizado na nitretação a plasma é relevante pois é necessário remover a camada passiva de óxido de cromo dos aços inoxidáveis. Em \cite{moller2001surface} discute-se que o \textit{sputtering}, além de remover a camada que impediria a penetração do átomos, também possui influência na cinética do processo pois a profundidade alcançada pela difusão que é usualmente função da raiz quadrada do tempo, com sputerring apresenta comportamento linear.

Na nitretação a gás diferentes métodos podem ser utilizados (químicos, mecânicos ou físicos) para preparar a superfície. Alguns estudos mostram que os fatores associados à ativação da superfície podem interferir em características da camada nitretada, como por exemplo, sua morfologia \cite{baranowska2010importance}. Em um estudo recente, Ömer estuda o efeito da cinética das reações na superfície no perfil de concentração de nitrogênio durante processos de nitretação, na qual observa que assim como para a nitretação gasosa, a nitretação a plasma pode ser caracterizada do ponto de vista de uma reação superficial, porém mais rápida do que no caso gasoso \cite{kucukyildiz2019thermo}.


Em \cite{tschiptschin2010estrutura}, conclui-se que a expansão da célula unitária na austenita (próximo de 10\%), provoca tensões residuais de compressão, que são responsáveis pelo endurecimento da superfície. Segundo os resultados de  Moskalioviene e Galdikas em \cite{moskalioviene2011stress}, com o aumento da tensão no reticulado, menor o coeficiente de difusão, ou seja, menor a influência do mecanismo dado pela Segunda Lei de Fick na difusão. 
Em \cite{kucukyildiz2019thermo}, o autor discute o efeito do gradiente de tensão, apontando que esta é uma força motriz adicional para a difusão e que aumenta a espessura da camada de difusão (como pode ser visto em \cite{li2017effect} - a difusão se extende para profundidades maiores, mas a concentração da superfície é reduzida) e argumenta-se que devido às altas tensões de compressão presentes na superfície com a introdução de nitrogênio, não é razoável desprezar seus efeitos. Sendo assim, as tensões de compressão presentes no fenômeno sendo estudado devem possuir influência relevante para entender melhor o comportamento observado e poderiam ser adicionados ao modelo do trabalho para melhorar a coerência dos resultados.

Com base nessas análises, possibilidades de melhorias do modelo desenvolvido seriam incorporar termos que considerem o efeito das tensões internas na difusão e utilizar diferentes modelamentos para a condição de contorno da superfície, levando em conta processos de preparação da superfície como \textit{sputtering}, ou outros efeitos relacionados à introdução do átomo como \textit{swelling}. Além disso, a utilização de outros valores como entrada para a simulação, como por exemplo energia de ativação, concentração de átomos e fator pré-exponencial de difusão, podem afetar os resultados das simulações ou podem servir para aplicá-las para materiais ou diferentes composições.
