Os primeros resultados obtidos para a simulação da Segunda Lei de Fick estavam de acordo com o esperado como pode ser visto pelas figuras das Seção \ref{sec:modelo11}. As curvas de concentração de nitrogênio em função da profundidade apresentaram um comportamento exponencial, com seu valor máximo na superfície, diminuindo quanto maior a distânica da superfície. Quanto maior o tempo, menor a inclinação da curva, esperado como um resultado do decorrer da difusão.

Para os resultados da Seção \ref{sec:modelo12}, na qual foi considerada que o efeito da nitretação gasosa na concentração superficial é que com o decorrer do tempo, a concentração tende àquela do equilíbrio, pode-se notar que o resultado obtido (ainda que correto conforme a Segunda Lei de Fick) não possui boa correlação com os resultados experimentais. Isso pode ser observado na Figura \ref{fig:csvar-gas2}, na qual para uma profundidade de aproximadamente 8$\mu m$ o perfil obtido experimentalmente apresenta uma queda brusca enquanto a solução dada pela Segunda Lei de Fick segue seu formato exponencial. Esse comportamento era esperado, dado que o pressuposto original era justamente que o mecanismo atuante na difusão do nitrogênio no aço não é dado pelo equacionamento de Fick.

Como pode ser visto na Figura \ref{fig:csvar-gas2}, a concentração de nitrogênio para profundidades maiores que uma dada distância, apresentaram concentração inferior à esperada pela solução da equação de difusão de Fick. Esse comportamento pode estar associado ao aprisionamento do átomos de nitrogênio, dado que na ocorrência desse fenômeno o número de interstícios ocupados na camada superficial é maior, reduzindo assim a probabilidade de saltos e o fluxo de átomos, causando um atraso na propagação da difusão para dentro do material.

Para a nitretação a plasma aplicada à Segunda Lei de Fick visto na Seção \ref{sec:modelo12}, os valores obtidos para a concentração superficial superaram os obtidas pelo mesmo modelo considerando nitretação a gás, ainda que parecem convergir para um valor após algumas horas de simulação. O valor obtido na superfície, próximo de 40\% após 22 horas, superestimam também os valores encontrados na literatura para a concentração de nitrogênio na austenita expandida - valor para o qual a possibilidade de precipitação deve ser considerada, outro motivo que não torna o resultado satisfatório para esse caso. Na Figura \ref{fig:csvar-plasma1}, os dados experimentais mostram uma penetração de nitrogênio para profundidades maiores do que mostram os resultados da simulação. Uma possível explicação para isso está na complexidade de modelar o processo de nitretação a plasma, pois este envolve diversos fatores que serão explorados em seguida.

As simulações para o modelo de \textit{trapping-detrapping}, apresentados na Seção \ref{sec:modelo2}, apresentaram o formato de curva esperado, no sentido que o perfil de concentração se inicia com uma leve inclinação, definida por uma região identificada normalmente como um platô, seguida de uma mudança brusca da inclinação indicando uma queda da concentração do átomo. Esse comportamento pode ser visto para a concentração superficial constante na Figura \ref{fig:td-cscte1} e na Figura \ref{fig:td-cscte-exp}, para concentração superficial em caso de nitretação gasosa na Figura \ref{fig:td-csvar-gas}. No caso da nitretação a plasma, esse resultado está representado na Figura \ref{fig:td-csvar-plasma}, porém suas características não ficaram muito visíveis para esse caso.

Na Figura \ref{fig:td-cscte-exp}, estão os resultados para alguns instantes da simulação, até 22 horas para concentração na superfície constante, e o resultado experimental apresentado por Christiansen e Somers em \cite{christiansen2008nitrogen}. Observa-se que as curvas parecem se aproximar da curva experimental ao longo do tempo, no entanto, o resultado final subestima a profundidade total obtida experimentalmente. Uma possível razão pode ser a ausência da influência das tensões de compressão no modelo, que promove um fluxo de difusão mais acelerado ao interior do material, como será discutido na Seção \ref{sec:discussao}.



As Figuras \ref{fig:td-cscte-both}, \ref{fig:td-csvar-gas-both} e \ref{fig:td-csvar-plasma-both} mostram o perfil de nitrogênio livre, nitrogênio aprisionado e nitrogênio total após 2 horas, para concentração superficial constante, para nitretação gasosa e nitretação a plasma, respectivamente. Nelas, observa-se que a concentração de nitrogênio em sítios de aprisionamentos é maior que a de nitrogênio livre e que a primeira possui uma queda mais brusca, enquanto a segunda possui um comportamento mais parecido com o da difusão clássica dada pela Segunda Lei de Fick. A somatória das duas concentrações resulta na curva com o comportamento conhecido da difusão de nitrogênio em aços, mencionada anteriormente. Destaca-se que os resultados estão dentro do esperado, pois o comportamento observado é o mesmo que foi obtido pelo modelo de Galdikas e Moskalioviene em \cite{moskalioviene2011modeling}. 

A Figura \ref{fig:td-csvar-gas-exp} mostra o decorrer da simulação para intervalos maiores de tempo e o resultado experimental de Somers para 23 horas de experimento. Pode-se observar que a concentração na superfície ultrapassa àquela do resultado que se espera obter, porém o restante do perfil de concentração parece condizente com a experiência.

Observa-se na Figura \ref{fig:td-csvar-gas-both} que a concentração de átomos em sítios de aprisionamento tende ao limite mais rapidamente que a concentração de nitrogênio livre para difundir. Inicialmente, o aumento da concentração de átomos aprisionados é maior que o de átomos aptos a difundir, o que demonstra que existe uma tendência maior dos átomos de ocupar os sítios do que de difundir. No entanto, conforme os sítios de cromo são ocupados, a tendência é que a concentração de nitrogênio livre comece a aumentar enquanto a outra concentração se mantém constante pois a possibilidade dos átomos estarem aprisionados é maior do que estarem livres, porém uma vez aprisionados é díficil deixarem essa posição. Esse comportamento é esperado pois os \textit{trap sites} tendem a ficar saturados com o tempo, logo, quanto maior a concentração total de nitrogênio, maior a concentração do mesmo livre para difundir.

As Figuras \ref{fig:td-cscte-compara} e \ref{fig:td-csvar-gas-compara} mostram como o perfil de concentração obtido se diferencia da Segunda Lei de Fick, principalmente devido ao platô no próximo da superfície.


Para a duração de 2 horas, os resultados obtidos para o perfil de concentração de nitrogênio no modelo de \textit{trapping-detrapping} considerando nitretação a plasma subestimou a concentração real obtida por \cite{moskalioviene2011modeling} como visto na Figura \ref{fig:td-csvar-plasma}. Pode-se notar que nenhum modelo obteve valores próximos daqueles obtidos por Moskalioviene e Galdikas para um intervalo de tempo pequeno como 2 horas. Possíveis explicações estão nas características do processo que não foram levadas em conta no modelo desse trabalho e também devido às simplificações feitas ao modelo, que serão discutidas na Seção \ref{sec:discussao}.

Uma simplificação feita para a simulação do modelo de \textit{trapping-detrapping} desse trabalho foi a omissão do fator 4$\pi$ dos coeficientes de \textit{trapping} e  \textit{detrapping}. Essa escolha pode ter trazido diferenças na velocidade com que a difusão ocorre para dentro do material, provocando divergências quando comparadas com outros resultados encontrados na literatura.

Outro fator relacionado à condição de contorno aplicada para a nitretação a plasma foi o valor da concentração de átomos hospedeiros na superfície. Devido à dificuldade de achar um valor, optou-se por utilizar um valor que proporcionasse um resultado próximo do esperado. Porém, dada a inconsistência dos resultados comparado aos dados experimentais e resultados de outros modelos obtidos na literatura, o modelamento dessa condição deve ser revisto. 


Com relação ao processo utilizado no artigo de Moskalioviene e Galdikas, observa-se que a nitretação a plasma foi feita sem a realização de sputtering na superfície. Segundo os autores, a energia de aceleração dos íons utilizada foi inferior a 15 eV, que é menor que o limite de \textit{sputtering} para a maioria dos materiais. Devido à ausência de \textit{sputtering}, observou-se um aumento de volume, para o qual foi medida uma taxa relativa a esse processo (chamada de \textit{swelling}). Para considerar os efeitos desse fenômeno, os autores optaram por adicionar um termo de \textit{swelling} ao modelo, o qual não foi considerado para a realização do modelo no trabalho em questão, logo pode representar mais uma fonte de divergência.