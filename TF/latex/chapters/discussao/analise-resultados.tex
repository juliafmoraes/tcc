Os primeros resultados obtidos para a simulação da Segunda Lei de Fick estavam de acordo com o esperado como pode ser visto pelas figuras das Seção \ref{sec:modelo11}. As curvas de concentração de nitrogênio em função da profundidade apresentaram um comportamento exponencial, com seu valor máximo na superfície, diminuindo quanto maior a distânica da superfície e quanto maior o tempo, menor a inclinação da curva, esperado como um resultado do decorrer da difusão.

Para os resultados da Seção \ref{sec:modelo12}, na qual foi considerado que o efeito da nitretação gasosa na concentração superficial é que com o decorrer do tempo, a concentração tende àquela do equilíbrio, pode-se notar que o resultado obtido (ainda que correto conforme a Segunda Lei de Fick) não possui boa correlação com os resultados experimentais. Isso pode ser observado na Figura \ref{fig:csvar-gas2}, na qual para uma profundidade de aproximadamente 8$\mu m$ o perfil obtido experimentalmente apresenta uma queda brusca enquanto a solução dada pela Segunda Lei de Fick segue seu formato exponencial. Esse comportamento era esperado, dado que o pressuposto original era justamente que o mecanismo atuante na difusão do nitrogênio no aço não é dado pelo equacionamento de Fick. Como pode ser visto na Figura \ref{fig:csvar-gas2}, devido ao potencial aprisionamento provocado pelos sítios de cromo, a concentração de nitrogênio para profundidades maiores que uma dada distância, apresentarão concentração inferior à esperada pela solução da equação de difusão de Fick.

Para a nitretação à plasma aplicada à Segunda Lei de Fick visto na Seção \ref{modelo22}, os valores obtidos para a concentração superficial superaram às obtidas pelo mesmo modelo considerando nitretação à gás, ainda que parecia convergir para um valor após algumas horas de simulação. Na Figura \ref{fig:csvar-plasma1}, os dados experimentais mostram uma penetração de nitrogênio para profundidades maiores do que mostram os resultados da simulação. Uma possível explicação para isso está na complexidade de modelar o processo de nitretação à plasma, pois este envolve diversos fatores que serão \myworries{??} explorados em seguida.

\myworries{nenhuma simulacao de trapdetrap rodou ate 22 horas }

As simulações para o modelo de trapping-detrapping apresentaram o formato de curva esperado, no sentido que o perfil de concentração se inicia com uma leve inclinação, definida por uma região identificada normalmente como um platô, seguida de uma mudança brusca da inclinação que indica uma queda da concentração do átomo.

\myworries{falta falar sobre o trappingdetrapping - basicamente todos}

\myworries{As profundidades obtidas estão de acordo com o esperado... ou nao }

Para a duração de 2 horas, os resultados obtidos para o perfil de concentração de nitrogênio no modelo de \textit{trapping-detrapping} considerando nitretação à plasma subestimou a concentração real obtida por \cite{moskalioviene2011modeling} como visto na Figura \myworries{colocar figura quando tiver}. Pode-se notar que nenhum modelo cheogu próximo dos valores obtidos por Moskalioviene e Galdikas para um intervalo de tempo pequeno como 2 horas. Possíveis explicações estão nas características do processo que não foram levadas em contas no modelo desse trabalho e também devido à simplificações feitas ao modelo.

Uma simplificação feita para a simulação do modelo de \textit{trapping-detrapping} desse trabalho foi a omissão do fator 4$\pi$ dos coeficientes de \textit{trapping} e  \textit{detrapping}. Essa escolha pode ter trazido diferenças na velocidade com que a difusão ocorre para dentro do material, provocando diferença nos resultados.

Com relação ao processo utilizado no artigo de Moskalioviene e Galdikas, observa-se que a nitretação à plasma foi feita sem a realização de sputtering na superfície. Segundo os autores a energia de aceleração dos íons utilizada foi inferior a 15 eV, que é menor que o limite de sputtering para a maioria dos materiais. Devido à ausência de sputtering, observou-se um aumento de volume, para o qual foi medido uma taxa relativa a esse processo (chamado de \textit{swelling}). Para considerar os efeitos desse fenômeno, os autores optaram por adicionar um termo de \textit{swelling} ao modelo, o qual não foi considerado para a realização do modelo no trabalho em questão, logo pode representar mais uma fonte de divergência.