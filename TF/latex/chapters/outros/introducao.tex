A resistência à corrosão dos aços inoxidáveis torna-os muito interessante para diversas aplicações e os faz serem amplamente utilizados em diversas indústrias como petroquímica, biomédica, alimentícia e nuclear \cite{peng2018numerical}. Entretanto, suas propriedades mecânicas, por exemplo sua dureza superficial e resistência, não são altas suficientes para aplicações abrasivas, com altas taxas de desgaste. Por esse motivo é interessante a obtenção de camadas superficiais nitretadas, que aumentam a dureza superficial e permitem que esses aços sejam utilizados em aplicações que exigem maior resistência ao desgaste \cite{moller2001surface}.

A descoberta dos aços supersaturados com nitrogênio ocorreu nos anos 1980. Inicialmente obtidos através de nitretação por plasma, passaram a ser estudados até que se concluiu que a presença desse elemento em quantidades colossais provocavam o surgimento de uma fase que passou a ser chamada de fase S, ou austenita expandida \cite{christiansen2006controlled}. No entanto, a obtenção dessa fase não é simples, uma vez que, para temperaturas maiores que 550°C, ocorre a precipitação de nitretos de cromo. Em função disso, cromo é retirado da solução sólida, o que reduz a resistência à corrosão do aço, uma consequência extremamente indesejada \cite{tschiptschin2010estrutura}.

Dessa forma, diferentes métodos de nitretação foram estudados e diversos modelos matemáticos têm sido criados para entender e prever o comportamento da difusão no nitrogênio nesses aços. Os modelos variam não apenas com relação ao processo de nitretação considerado, como também nos fatores que mais influenciam na difusão. Entre algumas das considerações dos modelos estão as tensões de compressão causada pela expansão do reticulado, a afinidade do cromo e do nitrogênio e efeitos de reações na superfície.

O objetivo do presente trabalho é buscar entender o perfil de concentração de átomos de nitrogênio em aços austeníticos após tratamentos de nitretação, mais especificamente estudar os motivos pelos quais o perfil encontrado não segue as leis clássicas de difusão - como a Segunda Lei de Fick - e criar um modelo matemático que permita simular computacionalmente a concentração do átomo no aço após o processo de nitretação. A ideia é buscar criar um modelo simples e verificar sua validade, ou seja, examinar se o problema pode ser modelado apenas com as considerações levantadas ou se é necessário adicionar outros fatores ao mesmo para se obter resultados satisfatórios.
Além desses objetivos, também buscou-se realizar um trabalho didático, que possa servir de base para futuros estudos na área.