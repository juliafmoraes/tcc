A utilização de aços inoxidáveis austeníticos está presente em diversas indústrias, como petroquímica, alimentícia, biomédica e nuclear. A resistência à corrosão apresentada pelos aços inoxidáveis, conferida pela presença de átomos de cromo, é uma das propriedades mais importantes para aplicações específicas desse material. No entanto, a baixa resistência mecânica à abrasão desses aços pode comprometer sua utilização em alguns casos. Por esse motivo, estuda-se a introdução de átomos de nitrogênio, que promovem uma expansão do reticulado, formando uma fase conhecida como austenita expandida, que aumenta a dureza da superfície. Diferentes processos de nitretação podem ser utilizados para obter tais resultados, porém eles devem ser realizados em temperaturas inferiores a 500°C, pois próximo dessa temperatura pode ocorrer a precipitação de nitretos de cromo, que retiram o cromo da matriz, responsável por impedir a oxidação desse material. Dessa forma, diferentes modelos foram criados para estudar a nitretação, que essencialmente é dada pela difusão do nitrogênio. Este trabalho teve como objetivo estudar os diferentes mecanismos que influenciam nessa difusão e outros modelos presentes na literatura, para propor um modelo simples que busca representar o perfil de concentração de nitrogênio obtido nos processos de nitretação. Considerou-se que a difusão era influenciada pelo mecanismo de aprisionamento de átomos de nitrogênio em sítios de cromo, devido à afinidade química entre os dois átomos, e duas condições de contornos foram testadas, uma para a nitretação a plasma e outra para a nitretação gasosa. O efeito do aprisionamento se provou condizente com experimentos e modelos presentes na literatura, porém as condições de contorno não corresponderam ao esperado em todos os testes.