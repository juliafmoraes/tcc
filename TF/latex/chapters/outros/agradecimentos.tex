Colocar em palavras os agradecimentos que gostaria de fazer ao refletir sobre os anos que antecederam a realização desse projeto não é simples. Começo agradecendo a minha família, que me proporcionou as mais diversas oportunidades que tornaram possível estudar na Escola Politécnica.

Agradeço especialmente todos aqueles que estiveram comigo durante esse período, nada seria possível sem vocês. Alguns estiveram comigo desde o ensino médio e me apoiaram até o último momento de produção desse trabalho, outros conheci nos primeiros anos e passamos os anos seguintes convivendo diariamente, durante altos e baixos, e mesmo longe pude continar contando com o apoio deles. Guardo um agradecimento especial para os que conheci nos útlimos anos desse período, que me receberam e acolheram tão bem em seus ambientes.
Ainde tive a sorte de conhecer pessoas fora da minha vida acadêmica que sempre me apoiaram e incentiveram meus estudos. 

A Escola Politécnica me porporcionou oportunidades pelas quais serei eternamente grata. Agradeço todos seus professores e funcionários que fazem parte desse percurso que construiu um pouco de quem sou hoje. 

Ao Professor Eduardo Franco de Monlevade, agradeço em especial pela paciência para responder minhas dúvidas e por concordar com minhas ideias e me apoiar durante todo o processo envolvido na execução desse trabalho durante o último ano.

Gostaria de agradecer também o Professor Fernando Akira Kurokawa, por me ajudar nas formulações matemáticas e na revisão do trabalho.

Alguns nomes que não podem deixar de ser mencionados: Lucas, Marina, Julia, Laura, João, Bruno, Pedro, Fabio, Gabriel, Renato, Fernanda, Ricardo, Borana, Luiza, Jacqueline, Amanda, Guilherme e Felipe.