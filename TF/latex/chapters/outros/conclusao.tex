Após analisar alguns modelos que buscam entender e prever os resultados da difusão de nitrogênio em diversos processos de nitretação de aços, foi possível criar um modelo simples a partir dos que foram utilizados para conhecer melhor o problema em questão e testar sua validade.

Analisando os resultados obtidos para o perfil de concentração de nitrogênio considerando diferentes condições de contorno, foi possível verificar a validade do modelo com relação ao comportamento da difusão no interior do material. Dessa forma, conclui-se que a afinidade entre o cromo e o nitrogênio que provoca um efeito de aprisionamento do nitrogênio em sítos de cromo, pode ser modelada matemáticamente considerando que o fenômeno segue cinética de primeira ordem.

Com relação às diferentes condições de contorno utilizadas, pode-se concluir que para a nitretação gasosa os resultados estão razoavelmente satisfatórios quando comparados aos experimentos e outros modelos da literatura. No entanto, para a nitretação a plasma, a utilização de determinados parâmetros e simplificações realizadas podem ter comprometido a obtenção de resultado mais próximos dos obtidos por outros autores. 

Dessa forma, levando em conta as limitações do modelo e os possíveis parâmetros envolvidos, são sugeridos possíveis aspectos que podem ser estudados mais profundamente e melhorados no modelo atual para que os resultados sejam mais próximos da realidade. Alguns desses aspectos são os parâmetros da condição de contorno para nitretação a plasma, o equacionamento dessa condição de contorno, os valores de entrada do modelo como as energias de ativação e coeficiente de difusão e os efeitos das tensões no processo de difusão.