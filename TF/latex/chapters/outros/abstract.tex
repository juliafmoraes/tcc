Austenitic stainless steels are currently employed in several industries, such as petrochemical, food, biomedical and nuclear. The corrosion resistance associated to stainless steels, which exists due to the presence of chromium atoms, is one of the most important properties for this material as it allows it to be used in specific situations. However, the low mechanical wear resistance of these steels may compromise their use in some cases. For this reason, the introduction of nitrogen atoms is studied. It leads to an expansion of the lattice, forming a phase known as expanded austenite, which increases the hardness of the surface. Different nitriding processes can be used to obtain these results, but they must be performed at temperatures below 500°C, because near this temperature it is possible that precipitation of chromium nitrides occurs, which removes the chromium from the matrix, responsible for preventing oxidation of this material. Thus, different models were created to study nitriding, which is essentially given by nitrogen diffusion. This study aimed to investigate the different mechanisms that influence this diffusion and other models present in the literature to propose a simple model that seeks to represent the nitrogen concentration profile obtained in nitriding processes. Diffusion was considered to be influenced by the trapping mechanism of nitrogen atoms at chromium sites due to the chemical affinity between the two atoms, and two boundary conditions were tested, one for plasma nitriding and one for gas nitriding. The effect of trapping proved to be consistent with experiments and models present in the literature, but the boundary conditions did not correspond to those expected in all tests.
