Os parâmetros utilizados estão explicitados na Tabela \ref{tab:resumoParametros}. Todos os valores de parâmetros que não são conhecidos da literatura foram retirados de \cite{moskalioviene2011modeling}.

\begin{table}[ht]
\centering
\setlength{\doublerulesep}{\arrayrulewidth}
{\def\arraystretch{2}\tabcolsep=10pt
\caption{Parâmetros compartilhados pelos modelos}
\resizebox{\textwidth}{!}{%
\begin{tabular}{c c c C{4cm}}
\hline\hline
Parâmetro & Descrição & Valor & Obs \\\hline
$E_A$ & Energia de ativação de Difusão & $1,7622 \times 10^{-19} J$  \\
$E_B$ & Energia de ativação de \textit{detrapping} & $4,48609 \times 10^{-20} J$  \\\
$D_0$ & Fator pré-exponencial de difusão & $8,37 \times 10^{-8} m^2/s$  \\
$H_0$  & Concentração de Átomos Hospedeiros& $H_0 = 4(1/R_t)^3 = 7,29 \times 10^{28} m^{-3}$ & o fator 4 aparece pois existem 4 átomos hospedeiros por célula unitária \cite{peng2018numerical} \\
$H_t$  & Concentração de Sítios de Aprisionamento & $H_t = 0.18H_0 = 1,31 \times 10^{28}m^{-3}$ \\ 
$k_B$  & Constante de Boltzmann & $ 1,38064852 \times 10^{-23} J/K$ \\ 
$R_t$     & Raio de aprisionamento           & $0,38 \times 10^{-9} m$  \\
\hline\hline
\end{tabular}
\label{tab:resumoParametros}
}
}
\end{table}

\subsection{Especificações do material}
Para a execução das simulações foi considerado que o aço em questão é um AISI 316L com composição 11,3 at.\%Ni, 19,5 at.\%Cr (e restante Fe, para efeito dos cálculos). Essa escolha foi feita pois foi o material utilizado por \cite{moskalioviene2011modeling} e \cite{christiansen2008nitrogen} usa um AISI 316 com composição similar.

\subsection{Unidades de concentração}
Na aplicação foram utilizadas as concentrações em átomos/m$^3$, mas para melhor entendimento dos resultados e conformidade com os artigos estudados, a concentração é mostrada em at.\%.
Para isso, foram utilizados os volumes molares dos elementos do material, definidos na \autoref{tab:volmolar} e o número de Avogadro ($N_A = 6,0221409 \times 10^{23}$).

\begin{table}[ht]
\centering
\setlength{\doublerulesep}{\arrayrulewidth}
{\def\arraystretch{2}\tabcolsep=10pt
\caption{Volume Molar para alguns elementos}
\begin{tabular}{c c c}
\hline\hline
Elemento & Valor & Fonte\\\hline
Fe & 7,09 $\times$ 10$^{-6}$ m$^3$/mol  & \cite{singman1984atomic}\\
Cr & 7,23 $\times$ 10$^{-6}$ m$^3$/mol  & \cite{singman1984atomic}\\
Ni & 6,59 $\times$ 10$^{-6}$ m$^3$/mol  & \cite{singman1984atomic}\\
%N & 4 $\times$ 10$^{-5}$ m$^3$/mol  & \cite{galdikas2011modeling}\\ 
\hline\hline
\end{tabular}
\label{tab:volmolar}
}
\end{table}

A partir desses volumes molares, é possível o obter a concentração em at.\% utilizando a concentração em m$^3$/mol.

\begin{table}[ht]
\centering
\setlength{\doublerulesep}{\arrayrulewidth}
{\def\arraystretch{2}\tabcolsep=10pt
\caption{Transformação de Unidade para concentração}
\resizebox{\textwidth}{!}{
\begin{tabular}{c c c c}
\hline\hline
Elemento & at.\% & Volume Molar (m$^3$/mol) & átomos/m$^3$\\\hline
Fe & 69,2\% & 7,09 $\times$ 10$^{-6}$  &  1,032  $\times$ 10$^{28}$\\
Cr & 19,5\% & 7,23 $\times$ 10$^{-6}$  &  1,624 $\times$ 10$^{28}$\\
Ni & 11,3\% & 6,59 $\times$ 10$^{-6}$  & 5,877  $\times$ 10$^{28}$\\
TOTAL (multiplicado pela composição) & &  & 8,534 $\times$ 10$^{28}$\\ 
\hline\hline
\end{tabular}
\label{tab:volmolar-convert}
}
}
\end{table}

Sendo assim, a composição de nitrogênio em at.\% pode ser obtida utilizando a eq.\autoref{eq:convert}

\begin{equation}
\label{eq:convert}
	N(at.\%) = \dfrac{N(m^{-3})}{N(m^{-3}) + 8,534 \times 10^{28}\\  } 
\end{equation}

