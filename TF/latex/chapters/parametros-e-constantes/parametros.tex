\FloatBarrier
\subsection{Parâmetros}
\label{params_sub}
Os parâmetros compartilhados por todos os modelos estão definidos na Tabela \ref{tab:resumoParametros}, e aqueles utilizados para o modelo de \textit{trapping-detrapping} estão explicitados na Tabela \ref{tab:resumoParametrosTD}. Todos os valores de parâmetros que não são conhecidos da literatura foram retirados de \cite{moskalioviene2011modeling}.

\begin{table}[!hbt]
\centering
\setlength{\doublerulesep}{\arrayrulewidth}
{\def\arraystretch{2}\tabcolsep=10pt
\caption{Parâmetros compartilhados pelos modelos}
\resizebox{\textwidth}{!}{%
\begin{tabular}{c c c}
\hline\hline
Parâmetro & Descrição & Valor \\\hline
$E_A$ & Energia de ativação de Difusão & $1,7622 \times 10^{-19} J$  \\
$D_0$ & Fator pré-exponencial de difusão & $8,37 \times 10^{-8} m^2/s$  \\
$k_B$  & Constante de Boltzmann & $ 1,38064852 \times 10^{-23} J/K$ \\ 
\hline\hline
\end{tabular}
\label{tab:resumoParametros}
}
}
\end{table}

\begin{table}[!hbt]
\centering
\setlength{\doublerulesep}{\arrayrulewidth}
{\def\arraystretch{2}\tabcolsep=10pt
\caption{Parâmetros para o modelo de \textit{trapping-detrapping}}
\resizebox{\textwidth}{!}{%
\begin{tabular}{c c c C{4cm}}
\hline\hline
Parâmetro & Descrição & Valor & Obs \\\hline
$E_B$ & Energia de ativação de \textit{detrapping} & $4,48609 \times 10^{-20} J$  \\\
$H_0$  & Concentração de Átomos Hospedeiros& $H_0 = 4(1/R_t)^3 = 7,29 \times 10^{28} m^{-3}$ & o fator 4 aparece pois existem 4 átomos hospedeiros por célula unitária \cite{peng2018numerical} \\
$H_t$  & Concentração de Sítios de Aprisionamento & $H_t = 0.18H_0 = 1,31 \times 10^{28}m^{-3}$ \\ 
$k_B$  & Constante de Boltzmann & $ 1,38064852 \times 10^{-23} J/K$ \\ 
$R_t$     & Raio de aprisionamento           & $0,38 \times 10^{-9} m$  \\
\hline\hline
\end{tabular}
\label{tab:resumoParametrosTD}
}
}
\end{table}

\FloatBarrier
\subsection{Especificações do material}
Para a execução das simulações foi considerado que o aço em questão é um AISI 316L com composição 11,3 at.\%Ni, 19,5 at.\%Cr (e restante Fe, para efeito dos cálculos). Essa escolha foi feita pois foi o material utilizado por \cite{moskalioviene2011modeling} e \cite{christiansen2008nitrogen} usa um AISI 316 com composição similar.
\FloatBarrier

\subsection{Unidades de concentração}
Na aplicação foram utilizadas as concentrações em átomos/m$^3$, mas para melhor entendimento dos resultados e conformidade com os artigos estudados, a concentração é mostrada em at.\%.
Para isso, foram utilizados os volumes molares dos elementos do material, definidos na \autoref{tab:volmolar} e o número de Avogadro ($N_A = 6,0221409 \times 10^{23}$).

\begin{table}[ht]
\centering
\setlength{\doublerulesep}{\arrayrulewidth}
{\def\arraystretch{2}\tabcolsep=10pt
\caption{Volume Molar para alguns elementos}
\begin{tabular}{c c c}
\hline\hline
Elemento & Valor & Fonte\\\hline
Fe & 7,09 $\times$ 10$^{-6}$ m$^3$/mol  & \cite{singman1984atomic}\\
Cr & 7,23 $\times$ 10$^{-6}$ m$^3$/mol  & \cite{singman1984atomic}\\
Ni & 6,59 $\times$ 10$^{-6}$ m$^3$/mol  & \cite{singman1984atomic}\\
%N & 4 $\times$ 10$^{-5}$ m$^3$/mol  & \cite{galdikas2011modeling}\\ 
\hline\hline
\end{tabular}
\label{tab:volmolar}
}
\end{table}

A partir desses volumes molares, é possível o obter a concentração em at.\% utilizando a concentração em m$^3$/mol.

\begin{table}[!ht]
\centering
\setlength{\doublerulesep}{\arrayrulewidth}
{\def\arraystretch{2}\tabcolsep=10pt
\caption{Transformação de Unidade para concentração}
\resizebox{\textwidth}{!}{
\begin{tabular}{c c c c}
\hline\hline
Elemento & at.\% & Volume Molar (m$^3$/mol) & átomos/m$^3$\\\hline
Fe & 69,2\% & 7,09 $\times$ 10$^{-6}$  &  1,032  $\times$ 10$^{28}$\\
Cr & 19,5\% & 7,23 $\times$ 10$^{-6}$  &  1,624 $\times$ 10$^{28}$\\
Ni & 11,3\% & 6,59 $\times$ 10$^{-6}$  & 5,877  $\times$ 10$^{28}$\\
TOTAL (multiplicado pela composição) & &  & 8,534 $\times$ 10$^{28}$\\ 
\hline\hline
\end{tabular}
\label{tab:volmolar-convert}
}
}
\end{table}

Sendo assim, a composição de nitrogênio em at.\% pode ser obtida utilizando a eq.\autoref{eq:convert}

\begin{equation}
\label{eq:convert}
	N(at.\%) = \dfrac{N(m^{-3})}{N(m^{-3}) + 8,534 \times 10^{28}\\  } 
\end{equation}

\subsection{Concentração de Equilíbrio}
\label{sec:ceq-param}
Para definir um valor para a concentração da superfície para implementação da solução com as condições de contorno descritas na Seção \autoref{sec:sol-numerica-2alei}, foi utilizado o artigo  \cite{christiansen2008nitrogen}, que possui o valor da ocupância de átomos de nitrogênio $y_N^{S}$ (mais especificamente a fração de interstícios octaédricos do reticulado cúbico de face centrada ocupada por nitrogênio) para o estado de equilíbrio, com $K_N=$2,49 $bar^{-1/2}$ (valor para o qual os dados experimentais do artigo em questão mostravam-se mais próximos dos dados do modelo), para 445°C (718K). A variável $K_N=\dfrac{p_{NH_3}}{p_{H_2}^{3/2}}$ é o potencial de nitretação para uma mistura gasosa de NH$_3$/H$_2$.

A partir do valor de $y_N^{eq}$, para uma temperatura de 718K, é possível definir a concentração em $mols/m^3$ de nitrogênio utilizando a relação apresentada no apêndice A2 do artigo \cite{jespersen2016modelling}, dada pela equação\autoref{eq:oc-to-m3}.

\begin{equation} \label{eq:oc-to-m3}
	C = \dfrac{4}{N_A} \cdot y_N \cdot \dfrac{1}{V(y_N)}
\end{equation}

O fator 4 é o número de interstícios octaédricos de uma célula unitária CFC, $N_A$ é o número de Avogadro ($N_A = 6,0221409 \times 10^{23} (mol^{-1})$) e $V(y_N)$ é o volume da célula unitária (em $m^3$) para uma dada ocupância de nitrogênio.

A função $V(y_N)$ foi aproximada por uma função linear: 
\begin{gather*}
	V(y_N) = 2,8147 \times 10^{-29} \cdot y_N + 4,7134 \times 10^{-29}
\end{gather*}

Sendo assim, pode-se reescrever a equação \autoref{eq:oc-to-m3} como:

\begin{equation} \label{eq:oc-to-m3-full}
	C = \dfrac{4}{6,0221409 \times 10^{23}} \cdot \dfrac{y_N}{2,8147 \times 10^{-29} \cdot y_N + 4,7134 \times 10^{-29}}
\end{equation}

Dessa forma, para as condições especificadas acima, o valor de $y_N$ é 0,28 e a concentrção é 33.805 mol/m$^3$ (2,0358 $\times$ 10$^{28}$ m$^{-3}$).

\begin{table}[ht]
\centering
\setlength{\doublerulesep}{\arrayrulewidth}
{\def\arraystretch{2}\tabcolsep=10pt
\caption{Parâmetros para concentração na superfície constante}
\resizebox{\textwidth}{!}{
\begin{tabular}{c c c}
\hline\hline
Parâmetro & Descrição & Valor \\\hline

$C_{eq}$ & Concentração de equilíbrio entre gás e aço &  47.318,64 mol/m$^3$ (2,0358 $\times$ 10$^{28}$ m$^{-3}$)   \\
\hline\hline
\end{tabular}
\label{tab:parametros_csvar}
}
}
\end{table}


\subsection{Paramêtros para Nitretação Gasosa}
\label{sec:param-gas}
Para simular a concentração na superfície durante nitretação gasosa de acordo com a descrição dada na Seção \autoref{sec:nit-gas}, foram utilizados os paramêtros dados pela \autoref{tab:parametros_csvar}.

\begin{table}[!htb]
\centering
\setlength{\doublerulesep}{\arrayrulewidth}
{\def\arraystretch{2}\tabcolsep=10pt
\caption{Parâmetros para concentração na superfície variável - nitretação gasosa}
\resizebox{\textwidth}{!}{
\begin{tabular}{c c c}
\hline\hline
Parâmetro & Descrição & Valor \\\hline
$\beta$ & Coeficiente para relacionar velocidade em que se atinge o equilíbrio & 0,0001 \\
$C_{eq}$ & Concentração de equilíbrio entre gás e aço &  47.318,64 mol/m$^3$ (2,0358 $\times$ 10$^{28}$ m$^-3$)    \\
\hline\hline
\end{tabular}
\label{tab:parametros_csvar}
}
}
\end{table}

Os valores da \autoref{tab:parametros_csvar} foram retirados de \cite{christiansen2008nitrogen}. O valor de $\beta$ foi definido testando o melhor \textit{fit} em dados experimentais e $C_{eq}$ foi definido como descrito na seção \autoref{sec:ceq-param}.


\subsection{Paramêtros para Nitretação à Plasma}
\label{sec:param-plasma}

Para a densidade de corrente média ($j$) da Seção \ref{sec:nit-plasma}, foi utilizado o valor encontrados no artigo \cite{galdikas2011modeling}, de 0,44mA/cm$^{2}$. 
